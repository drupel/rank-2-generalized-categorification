\documentclass{amsart}
\usepackage{amsbsy,amsmath,amssymb,latexsym,color}
\usepackage[margin=1.5in]{geometry}
\usepackage{hyperref}
\usepackage{xy}
\usepackage{tikz}
\usepackage{tikz-cd}

\usepackage[draft]{say}
\newcommand{\sayDR}[1]{\say[DR]{\color{violet}{\bf DR:}\;#1}}
\newcommand{\sayGT}[1]{\say[GT]{\color{blue}{\bf GT:}\;#1}}
\newcommand{\sayKI}[1]{\say[KI]{\color{red}{\bf KI:}\;#1}}

\input xy
\xyoption{all}

\newcommand{\margin}[1]{\marginpar{\tiny #1}}


\numberwithin{equation}{section}

\newtheorem{theorem}{Theorem}[section]
\newtheorem{proposition}[theorem]{Proposition}
\newtheorem{conjecture}[theorem]{Conjecture}
\newtheorem{corollary}[theorem]{Corollary}
\newtheorem{lemma}[theorem]{Lemma}

\theoremstyle{definition}

\newtheorem{remark}[theorem]{Remark}
\newtheorem{example}[theorem]{Example}
\newtheorem{definition}[theorem]{Definition}
\newtheorem{problem}[theorem]{Problem}


\newtheorem{algorithm}[theorem]{Algorithm}
\newtheorem{question}[theorem]{Question}



\def\CC{\mathbb{C}}
\def\FF{\mathbb{F}}
\def\JJ{\mathbb{J}}
\def\KK{\mathbb{K}}
\def\QQ{\mathbb{Q}}
\def\RR{\mathbb{R}}
\def\SS{\mathbb{S}}
\def\TT{\mathbb{T}}
\def\ZZ{\mathbb{Z}}

\def\bfd{\mathbf{d}}
\def\bfe{\mathbf{e}}
\def\ubfe{{\underline{\mathbf{e}}}}
\def\bff{\mathbf{f}}
\def\bfj{\mathbf{j}}
\def\bfk{\mathbf{k}}
\def\bfl{{\ensuremath{\boldsymbol\ell}}}
\def\bfM{\mathbf{M}}
\def\bfm{\mathbf{m}}
\def\bfN{\mathbf{N}}
\def\bfn{\mathbf{n}}
\def\bfp{\mathbf{p}}
\def\bfq{\mathbf{q}}
\def\bfr{\mathbf{r}}
\def\bfs{\mathbf{s}}
\def\bfV{\mathbf{V}}
\def\bfv{\mathbf{v}}
\def\bfX{\mathbf{X}}
\def\bfx{\mathbf{x}}

\def\cA{\mathcal{A}}
\def\cC{\mathcal{C}}
\def\cF{\mathcal{F}}
\def\cG{\mathcal{G}}
\def\cI{\mathcal{I}}
\def\cJ{\mathcal{J}}
\def\cK{\mathcal{K}}
\def\cP{\mathcal{P}}
\def\cQ{\mathcal{Q}}
\def\cR{\mathcal{R}}
\def\cS{\mathcal{S}}
\def\cT{\mathcal{T}}
\def\cX{\mathcal{X}}

\def\fJ{\mathfrak{J}}


\def\kk{\Bbbk}

\def\dim{\operatorname{dim}}
\def\injdim{\operatorname{inj.dim}}
\def\projdim{\operatorname{proj.dim}}
\def\frank{\operatorname{frank}}
\def\rank{\operatorname{rank}}
\def\wdim{\operatorname{wdim}}
\def\udim{\underline{\operatorname{dim}}}
\def\uiso{\underline{\operatorname{iso}}}
\def\urank{\underline{\operatorname{rank}}}
\def\uwdim{\underline{\operatorname{wdim}}}
\def\mod{\operatorname{mod}}
\def\gcd{\operatorname{gcd}}
\def\Hom{\operatorname{Hom}}
\def\im{\operatorname{im}}
\def\IHom{\operatorname{InjHom}}
\def\Ext{\operatorname{Ext}}
\def\rep{\operatorname{rep}}
\def\soc{\operatorname{soc}}
\def\coker{\operatorname{coker}}
\newcommand{\partition}{\vdash}
\newcommand{\into}{\hookrightarrow}
\newcommand{\onto}{\to\!\!\!\!\!\to}
\newcommand{\from}{\leftarrow}
\newcommand{\rev}[1]{\overline{#1}}%{{#1}^\dagger}
\newcommand{\Vect}{\text{Vect}}

\def\half{\frac{1}{2}}

\renewcommand{\eqref}[1]{{\rm (\ref{#1})}}
\newcommand{\rem}[1]{\left\langle#1\right\rangle}

\newcommand{\erase}[1]{{}}
\newcommand{\comment}[1]{{\marginpar{\LARGE}\bf(COMMENT: #1)}}
\newcommand{\shortnote}[1]{\fcolorbox{black}{yellow}{\color{black} #1}}

\title{Categorification of Rank 2 Generalized Cluster Algebras}


\begin{document}

% subsection

%\newpage
%-----------------------------------------------------------------------------------
%            sub section Filtered Hom
%-----------------------------------------------------------------------------------

\subsection{filtered morphisms}

We introduce the notion of filtered morphisms and use them to compute the dimensions of Grassmannians at vertices and quiver Grassmannians. We go first to the case of $Gr_\bfk(\JJ_d^n)$.

Let $K,L$ be $\JJ_d$ modules of type $\bfk=k_1\beta_1+\cdots+k_d\beta_d$ and $\ell=\ell_1\beta_1+\cdots+\ell_d\beta_d$. Then 
\[
\Hom_{\JJ_d}(K,L)=\bigoplus_{i,j}\Hom_{\JJ_d}(\JJ_i^{k_i},\JJ_j^{\ell_j})
\]
with dimension $\sum_{i,j} k_i\ell_j\min(i,j)$. Since
\[
	Gr_\bfk(\JJ_d^n)=\IHom_{\JJ_d}(K,\JJ_d^n)/\IHom_{\JJ_d}(K,K),
\]
the tangent bundle of $Gr_\bfk(\JJ_d^n)$ at $K$ is 
\[
T_KGr_\bfk(\JJ_d^n)=\Hom_{\JJ_d}(K,\JJ_d^n)/\Hom_{\JJ_d}(K,K).
\]
Thus, we have:
\[
	\dim Gr_\bfk(\JJ_d^n)=\sum_i nik_i -\sum_{i,j}k_ik_j\min(i,j)
\]

\begin{definition}
By an \emph{admissible filtration} of a $\JJ_d$-module $L$, we mean a filtration $0=F_0L\subseteq F_1L\subseteq \cdots\subseteq F_dL=L$ so that $tF_kL\subseteq F_{k-1}L$ for each $k$. We usually use the \emph{socle filtration}: $F_{k+1}L/F_{k}L=soc (L/F_{k}L)$. For any epimorphism $f:M\to L$ and any admissible filtration of $M$, the \emph{coinduced filtration} of $L$ is given by $F_kL=f(F_kM)$. A short exact sequence $A\to B\to C$ will be called \emph{filtered exact} if $F_kA\to F_kB\to F_kC$ is exact for all $k$.
\end{definition}

It is easy to see that, for any monomorphism $f:A\to B$, the filtration on $A$ induced from the socle filtration of $B$ by $F_kA=f^{-1}(F_kB)$ coincides with the socle filtration of $A$.

\begin{definition}
By a \emph{filtered homomorphism} between filtered $\JJ_d$-modules we mean a homomorphism $f:M\to L$ so that $f(F_kM)\subset F_kL$ for all $k$. We denote the set of such homomorphisms by $F_\cdot \Hom(M,L)$.
\end{definition}

\begin{lemma}
Let $A\to B\to C$ be filtered exact and let $M$ be given the socle filtration. Then
\[
	0\to F_\cdot\Hom(M,A)\to F_\cdot\Hom(M,B)\to F_\cdot\Hom(M,C)\to0
\]
is exact.
\end{lemma}

\begin{proof}
This follows from the observation that $F_\cdot \Hom(\JJ_k,X)=F_kX$.
\end{proof}

\begin{lemma}
Let $K$ be a $\JJ_d$-submodule of $L$ of type $\bfk$. Then the tangent space to $Gr_\bfk(M)$ at $K$ is isomorphic to $F_\cdot \Hom_{\JJ_d}(K,M/K)$ where $M/K$ is given the filtration coinduced by the socle filtration of $M$.
\end{lemma}

\begin{proof}
Since $Gr_\bfk(M)=\IHom(K,M)/\IHom(K,K)$, its tangent space if the cokernel of $\Hom(K,K)\to \Hom(K,M)$ which is $F_\cdot\Hom(K,M/K)$ by the previous lemma.
\end{proof}

\begin{theorem}
Let $M$ be a $\JJ_d$-representation of the quiver $Q$ and let $U$ be a subrepresentation of $M$ with dimension vector $\bfe$. Then
\[
	T_UGr_\bfe (M)=F_\cdot \Hom_{\JJ_dQ}(U,M/U)
\]
\end{theorem}

\begin{proof}
$Gr_\bfe(M)$ is the subspace of $\prod_i Gr_{e_i}(M_i)$ consisting of $n$-tuples $(U_i\subseteq M_i)_i$, where $U_i$ has type $e_i$, so that each $M_\alpha:M_i\to M_j$ sends $U_i$ into $U_j$. So, the tangent space at $U$ is the subspace of 
\[
	\prod_i T_{U_i}Gr_{e_i}(M_i)=\prod_i F_\cdot\Hom_{\JJ_d}(U_i,M_i/U_i)
\]
consisting of $n$-tuples of filtered morphisms $(C_i:U_i\to M_i/U_i)$ so that, for any family of $\JJ_d$-liftings $\widetilde C_i:U_i\to M_i$ of $C_i$ to $M_i$, the following diagram commutes, modulo $\varepsilon^2$, for some $K$-linear map $\zeta:U_i\to U_j$:
\[
%\xymatrixrowsep{10pt}
\xymatrixcolsep{40pt}
\xymatrix{%begin xy matrix
U_i\ar[d]_{U_\alpha+\varepsilon \zeta_\alpha}\ar[r]^{f_i+\varepsilon \widetilde C_i} &
	M_i\ar[d]^{M_\alpha}\\
U_j \ar[r]^{f_j+\varepsilon \widetilde C_j}& 
	M_j
	}%end xy matrix
\]
In other words, 
\begin{equation}\label{eq: deformation of U in M}
M_\alpha\widetilde C_i=\widetilde C_jU_\alpha+f_j\zeta_\alpha
\end{equation}
\underline{Claim:} This condition is equivalent to the condition:
\begin{equation}\label{eq: equivalent eq for U in M}
N_\alpha C_i= C_jU_\alpha :
%\xymatrixrowsep{10pt}\xymatrixcolsep{10pt}
\xymatrix{%begin xy matrix
U_i\ar[d]_{U_\alpha}\ar[r]^(.4){ C_i} &
	M_i/U_i\ar[d]^{N_\alpha}\\
U_j \ar[r]^(.4){ C_j}& 
	M_j/U_j
	}%end xy matrix
\end{equation}
Proof: ($\Rightarrow$) Let $p_i:M_i\to M_i/U_i$. Then $C_i=p_i \widetilde C_i$ and $p_j f_j=0$. So,
\[
	N_\alpha C_i=N_\alpha p_i\widetilde C_i=p_jM_\alpha\widetilde C_i=p_j\widetilde C_jU_\alpha+0=C_jU_\alpha
\]
($\Leftarrow$) Given \eqref{eq: equivalent eq for U in M}, let $(\widetilde C_i:U_i\to M_i)_i$ be any collection of liftings given by the fact that $\Hom_{\JJ_d}(U_i,M_i)\to F_\cdot\Hom_{\JJ_d}(U_i,M_i/U_i)$ is surjective. Choose any vector space basis $\{x_k\}$ for $U_i$. Then 
\[
	N_\alpha\widetilde C_i(x_k)-\widetilde C_jU_\alpha(x_k)\in \ker p_j
\]
Since $\ker p_j=\im f_j$, there exist $y_k\in C_k$ so that $f_j(y_k)=N_\alpha\widetilde C_i(x_k)-\widetilde C_jU_\alpha(x_k)$. Let $\zeta(x_k)=y_k$. This defines a linear map $\zeta_\alpha:U_i\to U_j$. Then,
\[
	N_\alpha\widetilde C_i(x_k)-\widetilde C_jU_\alpha(x_k)=f_j\zeta_\alpha(x_k)
\]
Since this is true for all $x_k$, we obtain \eqref{eq: deformation of U in M}. This completes the proof of the claim. The theorem follows.
\end{proof}

%\newpage\section{Copy}
%\input{FilteredHom}


\end{document}